\chapter{Course Description}\label{course-description}


This course provides an introduction to the biology and classification
of plants, fungi, and animals, their tissues, organ systems, development
and reproduction. Four hours of lecture and a two-hour lab session are
required each week.

\section*{Learning Objectives}\label{learning-objectives}

\begin{itemize}
\item
  Understand and explain the evolutionary relationships between the
  major groups of fungi, plants and animals, and their relationships
  with their habitats, and with other groups of organisms.
\item
  Identify the appropriate external and internal structures and their
  functions in representative organisms.
\item
  Identify some of the distinguishing characteristics of the major
  fungi, plant and animal groups studied, and to be able to recognize
  and/or give examples of organisms belonging to these major groups.
\item
  Identify (through the microscope, and in diagrams) representative
  examples of the major somatic tissue groups, and to demonstrate
  understanding of their characteristics, functions, and location in the
  vertebrate organism.
\item
  Identify the basic anatomy of the different organ systems studied, and
  to understand the related basic physiology.
\item
  Recognize some of the factors which influence human health, and the
  causes and symptoms of some common human diseases.
\item
  Learn and demonstrate basic lab dissections.
\end{itemize}
